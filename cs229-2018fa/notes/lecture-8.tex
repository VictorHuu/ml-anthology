\documentclass{article}

\title{Lecture 8 Data Splits,Models\& Cross-Validation}
\author{Yuntao Hu}
\date{}

\begin{document}
\maketitle
\section{Bias/Variance}
\section{Regularization}
\[
\min \frac{1}{2} \sum_{i=1}^m {\|y^{(i)}-\theta^Tx^{(i)}\|}^2+\frac{\lambda}{2}{\|\theta\|}^2
\]
The latter term is called regularization.
Inuitively, when $\lambda$ is larger ,the model becomes less overfitting and more under-fitting(High Bias).
If we want to maximize the objective ,we need to minus the regularization term.
\begin{quote}
\textbf{Q:}Why doesn't SVM overfitting like crazy?

\textbf{A:}It turns out that optimization of objective is to maximize the marigin of SVM.

\textbf{Q:} Do we have to regularize per-element of the parameters?

\textbf{A:} Not really. In a text classifiction example: logistic regression w/ regularization tends to outperform the Naive Bayes.
If you do not use regularization ,the No. of examples is at least the order of the parameters that you want to fit.
Ok back to the original question: why don't we adopt $\sum_j \lambda_j \theta_j^2$ ? Because choosing n $\lambda_j$ is too costly.

\textbf{Q:} Why don't SVM machines suffer too badly?

\textbf{A:} Minimize the penalty $W$ . All of the class of functions seperate the data with a large marigin ,that class has low complexity formalized by VC dimension.Since it has low VC dimension ,it's quick simple and it's quick unlikely that they over-fit.So ,it's convenient that SVM has low No. of SVs.
\end{quote}
\begin{quote}
\textbf{VC Dimension:}If you can find a set of $n$ points, so that it can be shattered by the classifier (i.e. classify all possible $2n$
 labeling correctly) and you cannot find any set of $n+1$ points that can be shattered (i.e. for any set of $n+1$
 points there is at least one labeling order so that the classifier can not separate all points correctly), then the VC dimension is $n$.
\end{quote}
\section{Map Estimation}
Given a training set $S=\{x^{(i)},y^{(i)}\}$, $P(\theta|S)=\frac{P(S|\theta)P(\theta)}{P(S)}$,

$\arg \max_{\theta}P(\theta|S)=\arg \max_{\theta} P(S|\theta)P(\theta)=\arg\max_{\theta}\left(\prod_{i=1}^m P(y^{(i)},x^{(i)},\theta)\right)P(\theta)$

Assume that $\theta \sim \mathbf{N}(0,\tau^2I)$

There are 2 schools of statistics: Frequentist vs. Bayesian.The former develops MLE while the former develops MAP(Maximum A Posteriori)
They are equivalent.
\section{Data Splits}
\subsection{hold out cross validation}
10,000 examples:
e.g. $\theta_0+\theta_1x$ or $\theta_0+\theta_1x+\theta_2x^2$
or choose value $\tau,\lambda,C$
\begin{enumerate}
\item[(a)]train each model (option for degree of polynomial) on $S_{train}$,get some hypothesis $h_i$
\item[(b)]Measure error on $S_{dev}$ model with lowest error on $S_{dev}$
\item[(c)] Evaluate the algorithm on $S_{test}$ as test and report the error.
\end{enumerate}
What if we measure errors on $S_{train}$, it over-fit.

Historically, 7/3 or 6/2/2 splits is a rule of thumb if you don't have much datasets.
If you have a giant dataset, the percentage of test/dev you set is shrinking. But still ,test and dev datasets needs to be large enough to see the difference of perfs among algs.
\subsection{k fold cross validation}
dev set a.k.a. cross validation set

don't make any decision on test data set ,instead use it to track your team's performance.

For a small dataset, it's improper to train on only partial data.
We should round-robin,that is, for a k-fold dataset,for k rounds each, train on $k-1$ pieces and test on the remaining piece,and then average the error.
Outside the loop ,we can try several choices of degree of polynomials.
Notice that the final optional step is to re-fit on all the data.
\begin{quote}
\textbf{Q:} Why do we need train/dev/test datasets?They're redundant!
\textbf{A:}In ridge regression, we need to choose both the model parameters w and the regularization parameter lambda in order to balance bias and variance.
\begin{enumerate}
\item[(a)]If we use only the training data to both train the model and select $\lambda$, then $\lambda$ will be chosen to minimize the training error. Since the training error typically decreases as the model becomes more complex, this approach tends to select a very small $\lambda$ and leads to overfitting.

\item[(b)]On the other hand, if we estimate w for each $\lambda$ using the training set and directly evaluate performance on the test set in order to choose $\lambda$, then the test set is implicitly used for model selection. This causes the test error to be a biased estimate of the true generalization error. The issue is not that the test set is insufficiently representative.

\item[(c)]Therefore, a better approach is to split the dataset into training, validation (dev), and test sets. The training set is used to estimate $w$ for a given $\lambda$, the validation set is used to select the best $\lambda$, and the test set is used only once to evaluate the final model’s generalization performance.
\end{enumerate}
\end{quote}
\subsection{Leave out one validation}
If the dataset is even small,we can use leave-out-one validation.
As the name implies, in every round,we should pick one item out and train another 19 entries.
\begin{quote}
\textbf{Q:} Suppose that in a 10-fold CV ,there are 10 estimates,can we measure the variance on those 10 estimates?

\textbf{A:} It turns out that these 10 estimates are correlated because the training data among them are heavily overlapped.(Futhermore,refer to Michael Kearns)

\textbf{Q:} Are we going to use k-fold CV in deep learning?

\textbf{A:} It depends mainly on the scale of the dataset.

\textbf{Q\&A:} In k-fold CV, using F1-score is quite complicated.

\textbf{Q:} How do we sample the dataset in k-fold CV?
\textbf{A:} Usually it's randomly shuffled.If the training set and test set simulate different distribution, we should adopt other methods.
\end{quote}
\section{Feature Selection(Iterative Feature Selection,not PCA)}
You may suspect that many features are useless.

Try adding one feature i to $\mathbf{F}$ and see which single-feature affects most to improve the performance,then add it to the $\mathbf{F}$
It's an iterative approach.
\end{document}